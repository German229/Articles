\documentclass[a4paper,12pt]{article}

% --- Кодировка и язык ---
\usepackage[T2A]{fontenc}
\usepackage[utf8]{inputenc}
\usepackage[russian]{babel}

% --- Математика и теоремы ---
\usepackage{amsmath,amssymb,amsthm,mathtools}

% --- Оформление ---
\usepackage{geometry}
\geometry{margin=2.2cm}
\usepackage{setspace}
\onehalfspacing

% --- Списки, таблицы, алгоритмы ---
\usepackage{enumitem}
\usepackage{booktabs}
\usepackage{array}
\usepackage{algorithm}
\usepackage{algpseudocode}

% --- Графика и ссылки ---
\usepackage{graphicx}
\usepackage{hyperref}
\hypersetup{
  colorlinks=true,
  linkcolor=blue,
  citecolor=blue,
  urlcolor=blue
}

% --- Удобные макросы ---
\newcommand{\R}{\mathbb{R}}
\newcommand{\N}{\mathbb{N}}

\theoremstyle{definition}
\newtheorem{definition}{Определение}
\newtheorem{proposition}{Утверждение}
\newtheorem{remark}{Замечание}

\title{Задача коммивояжёра и маршрутирование с поворотными штрафами:\\
теоретический обзор, постановка и подход к сведению к ATSP/QTSP}
\author{Голод Герман\\
НИЯУ МИФИ}
\date{\today}

\begin{document}
\maketitle

\section{Введение}

\subsection{Актуальность}
Задача коммивояжёра (Traveling Salesman Problem, TSP) является базовой моделью комбинаторной оптимизации и теоретической информатики: она демонстрирует типичную структуру NP-трудных задач, стимулирует развитие методов целочисленного программирования, эвристик и метаэвристик, а также служит испытательным полигоном для алгоритмических идей (кандидатные множества, локальный поиск, разрежение графов, гибридные процедуры) \cite{garey_johnson_1979,lawler1985,applegate2006,rego2011}.

В прикладных задачах маршрутизации по улично-дорожной сети (логистика, робототехника, беспилотники) существенную роль играют \emph{повороты} (задержки на перекрёстках, ограничения по радиусу поворота, штрафы за манёвры). Классическая постановка TSP учитывает лишь стоимость перехода между двумя последовательными вершинами, тогда как стоимость реального движения часто зависит от \emph{тройки} последовательных вершин (или пары последовательных рёбер): «куда вошли в вершину» и «куда вышли» \cite{stanek2019,fischer2014}.

\subsection{Обзор литературы}
\textbf{Классическая TSP и точные методы.}
Классические результаты включают динамическое программирование Беллмана и подход Хелда--Карпа \cite{bellman1962,heldkarp1962}. NP-полнота евклидовой TSP показана Пападимитриу \cite{papadimitriou1977}, общий аппарат NP-полноты систематизирован в книге Гарея--Джонсона \cite{garey_johnson_1979}. Для метрической TSP известен алгоритм Христофидеса с аппроксимационным отношением $3/2$ \cite{christofides1976}. Для евклидовой TSP существует PTAS (в фиксированной размерности) \cite{arora1998,mitchell1999}.

\textbf{Практически эффективные эвристики и «state-of-the-art».}
Лин--Керниган и особенно его эффективная реализация ЛКХ (LKH) стали «золотым стандартом» для больших инстансов \cite{helsgaun2000,rego2011}. Сильный вклад внесён в инженерные решения и вычислительное исследование TSP/Concorde \cite{applegate2006}. Для экстремально больших размеров важны методы разрежения/кандидатных рёбер и ускорения: POPMUSIC и построение хороших кандидатных списков \cite{taillard2019}, а также линейлогарифмическая эвристика Таилларда \cite{taillard2022}. Отдельные направления демонстрируют масштабирование на миллионы узлов в ограниченное время (Santa Claus Challenge) \cite{mariescu2021}.

\textbf{Вариации с поворотными/квадратичными стоимостями.}
Модели, где стоимость зависит от пары последовательных рёбер (или тройки вершин), естественно приводят к \emph{квадратичной} структуре цели и к постановкам QTSP и близким (включая угловые версии) \cite{fischer2014,stanek2019}. Для таких моделей важно уметь:
(i) корректно сформулировать задачу (как QTSP/AngleTSP),
(ii) сводить её к (A)TSP или MILP,
(iii) применять зрелые решатели/эвристики (LKH, branch-and-cut) на трансформированной модели.

\subsection{Формулировка темы}
Тема работы: \emph{теоретический анализ классической задачи коммивояжёра и разработка теоретического подхода к вариации маршрутизации на графе с поворотными штрафами, включая корректную постановку как QTSP/ATSP и обсуждение алгоритмических схем (редукции, MILP, эвристики) на уровне идеи.}

\subsection{Аннотация}
В работе дан систематизированный теоретический обзор TSP: постановки, вычислительная сложность, аппроксимации и наиболее практично значимые семейства алгоритмов. Далее формализована вариация маршрутизации с поворотными штрафами как задача с зависимостью стоимости от тройки последовательных вершин, связанная с QTSP и угловыми модификациями. Предложен теоретический путь доведения вариации «граф с поворотами» до уровня научной постановки: (1) корректная математическая модель, (2) два взаимодополняющих направления решения --- MILP/branch-and-cut и редукции к ATSP с использованием зрелых эвристик (LKH/кандидатные множества), (3) анализ вычислительной сложности, ограничений и ожидаемого поведения на больших размерах.

\subsection{Ключевые слова}
Задача коммивояжёра; TSP; ATSP; QTSP; локальный поиск; Lin--Kernighan; LKH; POPMUSIC; поворотные штрафы; угловая TSP; целочисленное программирование; редукции.

\subsection{Методы исследования}
Используются:
\begin{itemize}[leftmargin=2em]
  \item формализация задач и доказательные рассуждения (связь моделей, редукции);
  \item анализ вычислительной сложности (асимптотика, NP-трудность);
  \item обзор и сравнительный анализ алгоритмических семейств (точные методы, аппроксимации, эвристики);
  \item построение теоретической схемы решения варианта с поворотами: MILP-модель с тройками и редукция к ATSP/QTSP-подходам.
\end{itemize}

\section{Основная часть}

\subsection{Главные мысли и предложение решения проблемы}

\subsubsection{1. Классическая постановка TSP и базовые классы}
\begin{definition}[Симметричная TSP (STSP)]
Дан полный неориентированный граф $G=(V,E)$, $|V|=n$, и функция стоимости $c:E\to \R_{\ge 0}$.
Требуется найти гамильтонов цикл минимальной стоимости:
\[
\min_{\tau \in \mathcal{H}(V)} \sum_{\{i,j\}\in \tau} c_{ij},
\]
где $\mathcal{H}(V)$ --- множество гамильтоновых циклов на $V$.
\end{definition}

\begin{definition}[Асимметричная TSP (ATSP)]
Дан полный ориентированный граф на $V$ и стоимости дуг $c_{ij}$.
Найти ориентированный гамильтонов цикл минимальной стоимости:
\[
\min_{\tau \in \mathcal{H}^{\rightarrow}(V)} \sum_{(i,j)\in \tau} c_{ij}.
\]
\end{definition}

\begin{definition}[Метрическая TSP]
Говорят, что STSP метрична, если стоимости удовлетворяют неравенству треугольника:
$c_{ik}\le c_{ij}+c_{jk}$ для всех $i,j,k$.
\end{definition}

\textbf{Комментарий.}
Метрический случай важен для аппроксимаций (например, алгоритм Христофидеса \cite{christofides1976}).
Для евклидовой TSP (точки в $\R^d$ и $c_{ij}=\|x_i-x_j\|$) известны PTAS \cite{arora1998,mitchell1999}.

\subsubsection{2. NP-трудность и границы применимости точных методов}
\begin{proposition}[NP-трудность]
Классическая TSP NP-трудна, а евклидова TSP NP-полна \cite{garey_johnson_1979,papadimitriou1977}.
\end{proposition}

\textbf{Следствие для практики.}
Для больших $n$ точные методы (DP \cite{bellman1962,heldkarp1962}, branch-and-cut) становятся применимы лишь при сильной структуре/разрежении и мощной инженерии \cite{applegate2006}. Поэтому основная практическая линия на масштабах $10^4$--$10^6+$ опирается на эвристики (LK/LKH, кандидаты, локальные улучшения, разрежение) \cite{helsgaun2000,rego2011,taillard2019,taillard2022,mariescu2021}.

\subsubsection{3. Современная «практическая теорема»: почему работает LKH}
Семейство Lin--Kernighan решает задачу через последовательность \emph{$k$-opt} перестроек тура, выбирая перспективные замены ребер. Ключевой технологический компонент --- \emph{кандидатные множества рёбер}: вместо полного $O(n^2)$ перебора рассматривается малое число «хороших» соседей для каждой вершины \cite{helsgaun2000,rego2011}. Эффективность LKH основана на:
\begin{itemize}[leftmargin=2em]
  \item качественных кандидатных множествах (альфа-меры, POPMUSIC и родственные подходы) \cite{helsgaun2000,taillard2019};
  \item агрессивных локальных улучшениях и аккуратной организации поиска;
  \item высокой повторяемости и комбинировании решений, что важно на больших $n$ \cite{rego2011,taillard2022,mariescu2021}.
\end{itemize}

\subsubsection{4. Постановка вариации: TSP на графе с поворотными штрафами}
Рассмотрим модель, в которой переходная стоимость зависит от \emph{двух последовательных перемещений}. Это естественно для дорожных сетей: стоимость «поворота» проявляется в вершине при смене направления.

\begin{definition}[Turn-TSP (стоимость по тройкам вершин)]
Пусть задан ориентированный или неориентированный граф движения $G=(V,E)$ и функция базовой стоимости рёбер $d:E\to \R_{\ge 0}$.
Пусть также задана функция штрафа за поворот
\[
p: V\times V\times V \to \R_{\ge 0}, \qquad p(i,j,k) \ \text{определяет штраф при переходе}\ i\to j\to k.
\]
Требуется найти гамильтонов цикл $\tau=(v_1,\dots,v_n,v_1)$, минимизирующий
\[
\min_{\tau} \left(
\sum_{t=1}^{n} d(v_t,v_{t+1})\;+\;\lambda \sum_{t=1}^{n} p(v_{t-1},v_t,v_{t+1})
\right),
\]
где индексы по модулю $n$, а $\lambda\ge 0$ --- коэффициент важности поворотов.
\end{definition}

\begin{remark}
Если $p\equiv 0$, получаем классическую TSP. Следовательно, Turn-TSP не проще TSP и как минимум NP-трудна.
\end{remark}

\textbf{Связь с QTSP.}
Такая цель имеет квадратичную/второго порядка структуру по последовательным рёбрам и естественно относится к классу QTSP и смежных постановок \cite{fischer2014}. Угловая версия (AngleTSP), где $p(i,j,k)$ соответствует углу поворота в вершине $j$, исследована как геометрическая вариация \cite{stanek2019}.

\subsubsection{5. Научно корректное доведение вариации до «решаемой» модели}

Далее предлагается теоретическая схема, которую можно развивать до полноценной научной статьи (и впоследствии --- до экспериментального исследования).

\paragraph{5.1. MILP-модель с переменными для троек}
Введём бинарные переменные:
\[
x_{ij}\in\{0,1\} \quad \text{(дуга/ребро $i\to j$ включено в тур)},\qquad
z_{ijk}\in\{0,1\} \quad \text{(в туре встречается фрагмент $i\to j\to k$)}.
\]

\textbf{Базовые ограничения тура (ATSP-стиль).}
\begin{align}
&\sum_{j\ne i} x_{ij} = 1,\quad \forall i\in V \qquad \text{(ровно один выход)},\\
&\sum_{j\ne i} x_{ji} = 1,\quad \forall i\in V \qquad \text{(ровно один вход)},\\
&\text{(ограничения против подтуров, например MTZ или cut'ы)}.
\end{align}

\textbf{Связь $z$ с $x$.}
Для всех различных $i,j,k$:
\begin{align}
&z_{ijk} \le x_{ij},\\
&z_{ijk} \le x_{jk},\\
&z_{ijk} \ge x_{ij}+x_{jk}-1.
\end{align}

\textbf{Целевая функция.}
\[
\min \sum_{i\ne j} d_{ij} x_{ij} + \lambda \sum_{i,j,k} p_{ijk}\, z_{ijk}.
\]

\textbf{Комментарий по научной новизне.}
Данная MILP-постановка сама по себе известна как стандартный путь линеаризации «стоимостей по тройкам». Научно значимое развитие здесь --- выбор:
\begin{itemize}[leftmargin=2em]
  \item класса ограничений (cutting planes) и нижних оценок;
  \item специальных структур $p_{ijk}$ (например, угловая геометрическая структура) \cite{stanek2019};
  \item схемы разрежения и построения кандидатов, чтобы уйти от полного $O(n^3)$ по тройкам.
\end{itemize}

\paragraph{5.2. Редукция Turn-TSP к QTSP/ATSP и использование зрелых эвристик}
Литература по QTSP демонстрирует, что квадратичная структура может быть сведена к классическим постановкам через полиномиальные преобразования и далее решаться стандартным ПО \cite{fischer2014}. Для практической линии (когда нужен масштаб) целесообразна архитектура:

\begin{enumerate}[leftmargin=2em]
  \item \textbf{Упорядочивание модели:} фиксировать $p(i,j,k)$ как локальную функцию поворота (угол/категория манёвра), допускающую быстрый доступ.
  \item \textbf{Разрежение:} строить для каждого $j$ кандидатные множества выходов $k\in Cand(j)$ и входов $i\in Cand^{-}(j)$, чтобы число потенциальных троек $(i,j,k)$ стало управляемым. Здесь применимы идеи кандидатных рёбер, POPMUSIC и ускоренного построения хороших ребёр \cite{helsgaun2000,taillard2019,taillard2022}.
  \item \textbf{Эвристика на редуцированном пространстве:} применить локальный поиск уровня LK/LKH на расширенном представлении (где «состояние» учитывает входящее ребро) либо использовать QTSP-ориентированные эвристики и матэвристики (подзадачи оптимизируются точно) \cite{fischer2014,stanek2019}.
\end{enumerate}

\paragraph{5.3. Концептуальная схема «расширения состояния» (edge-based state space)}
Интуитивно поворотная стоимость делает задачу \emph{второго порядка}: стоимость следующего шага зависит от предыдущего. Стандартный трюк в оптимизации на графах --- расширить пространство состояний.

\begin{definition}[Граф состояний по ориентированным рёбрам]
Пусть базовый граф движения имеет ориентированные дуги $(u\to v)$.
Введём состояние $s=(u\to v)$ как «мы приехали в $v$ из $u$».
Переход $s=(u\to v)\to s'=(v\to w)$ имеет стоимость $d_{vw}+\lambda p(u,v,w)$.
\end{definition}

Далее задача выбора тура превращается в задачу выбора цикла по состояниям с дополнительными ограничениями «посетить каждую вершину ровно один раз» (это приводит к постановкам семейства generalized TSP / ATSP на трансформированном графе). На уровне научной статьи здесь важно:
\begin{itemize}[leftmargin=2em]
  \item строго описать преобразование и доказать эквивалентность;
  \item оценить рост размерности после преобразования;
  \item выделить условия, при которых разрежение кандидатов делает подход вычислимо осуществимым.
\end{itemize}

\subsubsection{6. Теоретические оценки сложности и ожидаемое поведение}
\textbf{Базовая трудность.}
Turn-TSP NP-трудна уже потому, что включает TSP как частный случай ($p\equiv 0$).

\textbf{Размерность моделей.}
\begin{itemize}[leftmargin=2em]
  \item MILP с $x_{ij}$ имеет $O(n^2)$ переменных, а с тройками $z_{ijk}$ --- $O(n^3)$ в худшем случае, что требует разрежения.
  \item Расширение состояния по ориентированным рёбрам даёт число состояний порядка $O(|E|)$ (или $O(n^2)$ для полного графа), что может быть приемлемо только при сильном разрежении.
\end{itemize}

\textbf{Почему разрежение принципиально.}
Классические LKH/POPMUSIC идеи демонстрируют, что качество решения в значительной мере определяется качеством небольшого набора кандидатов \cite{helsgaun2000,taillard2019,taillard2022}. Поэтому «научно корректная» стратегия для Turn-TSP на больших $n$ --- построить кандидаты так, чтобы:
\[
\forall j:\ |Cand(j)| = O(\log n)\ \text{или константа, при сохранении качества тура.}
\]
Это переводит число учитываемых троек к $O(n\cdot |Cand^{-}(j)|\cdot |Cand(j)|)$, что становится практически приемлемым.

\subsubsection{7. Алгоритмическая схема (уровень идеи)}
Ниже приведена схема, которую можно развить до реализации и экспериментальной части (при необходимости).

\begin{algorithm}[H]
\caption{Turn-TSP на основе разрежения и локального поиска (идея)}
\begin{algpseudocode}[1]
\Require множество вершин $V$, базовые стоимости $d_{ij}$, штрафы $p_{ijk}$, параметр $\lambda$
\Ensure тур $\tau$
\State Построить кандидатные множества $Cand(j)$ (например, по $d_{ij}$ и/или по эвристике построения «хороших рёбер»)
\State Инициализировать тур $\tau_0$ (например, жадный nearest neighbor на $d_{ij}$, затем улучшить 2-opt по $d_{ij}$)
\State Определить функцию стоимости тура $F(\tau)=\sum d+\lambda \sum p$ по тройкам
\Repeat
  \State Выполнить локальное улучшение тура (обобщение $k$-opt), рассматривая перестройки, затрагивающие только рёбра из кандидатов
  \State Принять улучшение, если $F(\tau)$ уменьшилась
\Until{нет улучшений или достигнут лимит}
\State \Return $\tau$
\end{algpseudocode}
\end{algorithm}

\textbf{Что делает это «научным».}
Чтобы довести до уровня научной статьи, нужны:
\begin{itemize}[leftmargin=2em]
  \item строгое определение кандидатов и доказуемые/эмпирические аргументы, почему качество не деградирует слишком сильно;
  \item формальный анализ трудоёмкости одного шага локального поиска;
  \item сравнение с базовыми подходами: (i) MILP на малых $n$, (ii) эвристика без поворотных штрафов, (iii) специализированные QTSP/AngleTSP эвристики \cite{fischer2014,stanek2019}.
\end{itemize}

\subsubsection{8. Плюсы и минусы предложенного подхода}
\textbf{Плюсы:}
\begin{itemize}[leftmargin=2em]
  \item корректная математическая постановка (Turn-TSP как QTSP-подобная задача) \cite{fischer2014};
  \item совместимость с богатой экосистемой TSP-алгоритмов (LK/LKH, разрежение, кандидаты) \cite{helsgaun2000,rego2011,taillard2019,taillard2022};
  \item естественная интерпретация для транспортных сетей и робототехники (штрафы за манёвр).
\end{itemize}

\textbf{Минусы/риски:}
\begin{itemize}[leftmargin=2em]
  \item рост размерности при учёте троек (теоретически $O(n^3)$);
  \item необходимость аккуратно строить кандидаты и вычислять $p_{ijk}$;
  \item возможная асимметрия и «неметричность» стоимости, что осложняет аппроксимационные гарантии.
\end{itemize}

\section{Завершение}

\subsection{Выводы}
\begin{enumerate}[leftmargin=2em]
  \item TSP остаётся центральной задачей комбинаторной оптимизации; теоретические результаты (NP-полнота, аппроксимации, PTAS для евклидова случая) формируют границы применимости \cite{garey_johnson_1979,papadimitriou1977,christofides1976,arora1998,mitchell1999}.
  \item Практическая эффективность на больших размерах обеспечивается эвристиками семейства Lin--Kernighan и инженерными приёмами разрежения/кандидатов \cite{helsgaun2000,rego2011,taillard2019,taillard2022,mariescu2021}.
  \item Вариация «граф с поворотами» естественно формализуется как задача со стоимостью по тройкам вершин и связана с QTSP/AngleTSP \cite{fischer2014,stanek2019}. Это позволяет строго обосновать NP-трудность и корректно выбрать инструментарий.
  \item Предложен теоретический путь доведения до научного уровня: (i) MILP-модель с линеаризацией троек, (ii) редукции/расширение состояния и применение зрелых эвристик с разрежением, (iii) анализ сложности и рисков.
\end{enumerate}

\subsection{Список литературы}
\begin{thebibliography}{99}

\bibitem{bellman1962}
R.~Bellman.
\newblock Dynamic programming treatment of the travelling salesman problem.
\newblock \emph{Journal of the ACM}, 9(1):61--63, 1962.

\bibitem{heldkarp1962}
M.~Held and R.~M. Karp.
\newblock A dynamic programming approach to sequencing problems.
\newblock \emph{Journal of the Society for Industrial and Applied Mathematics}, 10(1):196--210, 1962.

\bibitem{garey_johnson_1979}
M.~R. Garey and D.~S. Johnson.
\newblock \emph{Computers and Intractability: A Guide to the Theory of NP-Completeness}.
\newblock W.\ H.\ Freeman, 1979.

\bibitem{papadimitriou1977}
C.~H. Papadimitriou.
\newblock The Euclidean travelling salesman problem is NP-complete.
\newblock \emph{Theoretical Computer Science}, 4(3):237--244, 1977.

\bibitem{christofides1976}
N.~Christofides.
\newblock Worst-case analysis of a new heuristic for the travelling salesman problem.
\newblock Technical Report 388, Graduate School of Industrial Administration, Carnegie Mellon University, 1976.

\bibitem{arora1998}
S.~Arora.
\newblock Polynomial time approximation schemes for Euclidean traveling salesman and other geometric problems.
\newblock \emph{Journal of the ACM}, 45(5):753--782, 1998.
\newblock DOI: \href{https://doi.org/10.1145/290179.290180}{10.1145/290179.290180}.

\bibitem{mitchell1999}
J.~S.~B. Mitchell.
\newblock Guillotine subdivisions approximate polygonal subdivisions: A simple polynomial-time approximation scheme for geometric TSP, $k$-MST, and related problems.
\newblock \emph{SIAM Journal on Computing}, 28(4):1298--1309, 1999.
\newblock DOI: \href{https://doi.org/10.1137/S0097539796309764}{10.1137/S0097539796309764}.

\bibitem{lawler1985}
E.~L. Lawler, J.~K. Lenstra, A.~H.~G. Rinnooy Kan, and D.~B. Shmoys (eds.).
\newblock \emph{The Traveling Salesman Problem: A Guided Tour of Combinatorial Optimization}.
\newblock Wiley, 1985.

\bibitem{applegate2006}
D.~L. Applegate, R.~E. Bixby, V.~Chv{\'a}tal, and W.~J. Cook.
\newblock \emph{The Traveling Salesman Problem: A Computational Study}.
\newblock Princeton University Press, 2006.

\bibitem{helsgaun2000}
K.~Helsgaun.
\newblock An effective implementation of the Lin--Kernighan traveling salesman heuristic.
\newblock \emph{European Journal of Operational Research}, 126(1):106--130, 2000.

\bibitem{rego2011}
C.~Rego, D.~Gamboa, F.~Glover, and C.~Osterman.
\newblock Traveling salesman problem heuristics: Leading methods, implementations and latest advances.
\newblock \emph{European Journal of Operational Research}, 211(3):427--441, 2011.

\bibitem{taillard2019}
{\'E}.~D. Taillard and K.~Helsgaun.
\newblock POPMUSIC for the travelling salesman problem.
\newblock \emph{European Journal of Operational Research}, 272(2):420--429, 2019.
\newblock DOI: \href{https://doi.org/10.1016/j.ejor.2018.06.039}{10.1016/j.ejor.2018.06.039}.

\bibitem{taillard2022}
{\'E}.~D. Taillard.
\newblock A linearithmic heuristic for the travelling salesman problem.
\newblock \emph{European Journal of Operational Research}, 297(2):442--450, 2022.
\newblock DOI: \href{https://doi.org/10.1016/j.ejor.2021.05.034}{10.1016/j.ejor.2021.05.034}.

\bibitem{skinderowicz2022}
R.~Skinderowicz.
\newblock Improving Ant Colony Optimization efficiency for solving large TSP instances.
\newblock \emph{Applied Soft Computing}, 120:108653, 2022.
\newblock DOI: \href{https://doi.org/10.1016/j.asoc.2022.108653}{10.1016/j.asoc.2022.108653}.

\bibitem{mariescu2021}
R.~Mariescu-Istodor and P.~Fr{\"a}nti.
\newblock Solving the Large-Scale TSP Problem in 1 h: Santa Claus Challenge 2020.
\newblock \emph{Frontiers in Robotics and AI}, 8:689908, 2021.
\newblock DOI: \href{https://doi.org/10.3389/frobt.2021.689908}{10.3389/frobt.2021.689908}.

\bibitem{fischer2014}
A.~Fischer, F.~Fischer, G.~J{\"a}ger, J.~Keilwagen, P.~Molitor, and I.~Grosse.
\newblock Exact algorithms and heuristics for the Quadratic Traveling Salesman Problem with an application in bioinformatics.
\newblock \emph{Discrete Applied Mathematics}, 166:97--114, 2014.
\newblock DOI: \href{https://doi.org/10.1016/j.dam.2013.09.011}{10.1016/j.dam.2013.09.011}.

\bibitem{stanek2019}
R.~Stan{\v e}k, P.~Greistorfer, K.~Ladner, and U.~Pferschy.
\newblock Geometric and LP-based heuristics for angular travelling salesman problems in the plane.
\newblock \emph{Computers \& Operations Research}, 108:97--111, 2019.
\newblock DOI: \href{https://doi.org/10.1016/j.cor.2019.01.016}{10.1016/j.cor.2019.01.016}.

\bibitem{ortools_cite}
Google OR-Tools.
\newblock How to cite OR-Tools and its solvers (routing library).
\newblock \emph{Developers documentation}, accessed 2025.
\newblock URL: \href{https://developers.google.com/optimization/support/cite}{developers.google.com/optimization/support/cite}.

\end{thebibliography}

\end{document}
